% Comentarios
\documentclass{scrbook}

% Para acentos chéveres
\usepackage[utf8]{inputenc}
\usepackage[T1]{fontenc}
\usepackage{amsmath,amsthm}

\newtheorem{theorem}{Teorema}
\newtheorem{lemma}[theorem]{Lema}
\newtheorem{corolario}{Corolario}

\theoremstyle{definition}

\newtheorem{definicion}{Definición}[chapter]
\newtheorem{obs}{Observación}

\DeclareMathOperator{\sen}{sen}
\newcommand{\R}{\mathbb{R}}

\begin{document}

\part{Primera parte}\label{part:etiqueta}

\chapter{Un título}\label{cap:etiqueta} %Comentarios
El porcentaje \% no se debe quitar. % Este sí.

Y tenemos que $x<\epsilon$ o $y>2$.

\textbf{sdflgk}

\textit{sdgethw fj dfg erg sd}

\section{Una sección}\label{sec:etiquetaSec1}

\begin{theorem}\label{teo:etiquetaTeo}
Este es el contenido de un teorema.
\begin{equation*}
	F(x)=\int_1^\pi \frac{2x^2 - \sen(e^{i\pi})}{\cos(1/x)}\,dx
\end{equation*}
\end{theorem}

\begin{lemma}\label{lema:etiquetaLema}
Un lema.
\end{lemma}

\begin{corolario}\label{coro:etiquetaCoro}
Un corolario.
\end{corolario}

\begin{definicion}\label{def:etiquetaDef}
Una definición con estilo menos enfático.
\end{definicion}




\section{Otra}\label{sec:Sec3}

Van unas listas relacionadas con la sección~\ref{sec:etiquetaSec1}
y el capítulo \ref{cap:etiqueta}

\begin{enumerate}
\item Un elemento

\item Un elemento

\item Un elemento
\item Un elemento

\end{enumerate}

\begin{itemize}
\item Un elemento

\item Un elemento

\item Un elemento
\item Un elemento

\end{itemize}


\section{Una sección}\label{sec:etiquetaSec2}

\end{document}
