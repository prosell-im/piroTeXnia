%La figura~\ref*{fig:primerafigura} muestra algunas líneas de la imagen-ejemplo.

\subsection{Table (\texttt{table})}\label{table}
Las tablas se usan para estructurar datos en filas y columnas. Se pueden personalizar ajustando alineaciones y bordes.

Para crear un índice con todas las figuras del documento, se utiliza \verb|\listoftables|.\\

\begin{table}[h]
    \centering
    \begin{tabular}{c|c|l}
        \hline
        Parámetro & Significado & Descripción\\
        \hline
        l & left & Alinea el texto a la izquierda de la celda \\
        \hline
        c & center & Centra el texto de la celda\\
        \hline
        r & right & Alinea el texto a la derecha de la celda\\
        \hline
    \end{tabular}
    \caption{Parámetros para controlar la posición del texto en una tabla.}
    \label{tabla:parametros_tabla}
\end{table}

\subsubsection*{Sintaxis:}
\begin{codigo}
\begin{LTXexample}[numbers=none]
    \begin{table}[placement]
        \centering
        \begin{tabular}{|c|c|c|}
        \hline
        Columna 1 & Columna 2 & Columna 3 \\ \hline
        Dato 1 & Dato 2 & Dato 3 \\ \hline
    \end{tabular}
    \caption{Título de la tabla.}
    \label{unatabla}
    \end{table}
\end{LTXexample}
\label{sintaxis:table}
\end{codigo}

\subsubsection*{Ejemplos:}

\begin{itemize}
    \item Usando la imagen 
    \begin{codigo}
    \begin{LTXexample}[numbers=none]
        \begin{table}[h]
            \centering
            \begin{tabular}{lcrp{3cm}}
            \toprule
            Color & Forma & Posición & Nombre \\ \midrule
            Azul & Recta & Dentro & Diámetro\\ 
            Rojo & Recta & Dentro & Radio\\
            Verde & Curva & Sobre & Arco\\
            \bottomrule
            \end{tabular} 
        \caption{Título de la tabla.}
        \label{tabla:unatabla}
        \end{table}
    \end{LTXexample}
    \label{ejemplo:table}
    \end{codigo}

\end{itemize}

\section{Buenas prácticas}
\begin{enumerate}
    \item Aunque  ambos entornos acceptan imágenes y texto, usa `figure` para imágenes y diagramas, y `table` para datos tabulados.
    \item Siempre incluye etiquetas (`\verb|\label|`) para facilitar las referencias cruzadas.
    \item Para controla con mayor precisión la posición de los flotantes utiliza el paquete \hyperref[ch:float]{\texttt{float}}.
\end{enumerate}


\end{document}
