\documentclass{amsbook}
\usepackage{preamble}


\title{Manual de \LaTeX}

\begin{document}
\maketitle

\chapter{Conceptos Básicos}

\section{Lenguajes de Marcado y \LaTeX}

Un lenguaje de marcado es un lenguaje que permite codificar el formato y estructura de un documento a través de comandos escritos a la par del texto en un archivo conocido como \wrn{archivo fuente}. Después, el archivo fuente se procesa y se crea un documento que ya tiene el formato y estructura aplicados al texto. 

En particular, \LaTeX es un lenguaje de marcado. Con él, el formato y estructura se especifican a través de \wrn{comandos} y \wrn{ambientes}. Un comando básico tiene la siguiente forma

\begin{latexcode}
\comando{param}
\end{latexcode}

donde \latexinline{comando} es el nombre del comando y \latexinline{param} es un parámetro. Los comandos pueden tener más parámetros (\latexinline{\comando{param1}{param2}...}), parámetros opcionales especificados entre \([\) y \(]\) (\latexinline{\comando[opc]{param}}), e incluso pueden no tener parámetros (\latexinline{\comando}). Los ambientes, por otro lado, tienen la siguiente forma

\begin{latexcode}
\begin{ambiente}
param
\end{ambiente}
\end{latexcode}

\section{Comandos y Ambientes}



\section{Documento Minimal}

Empezamos creando un directorio y un archivo fuente con extensión \latexinline{.tex}. Debe contener como mínimo el tipo de documento, especificado con el comando \latexinline{\documentclass{}}, y el ambiente \latexinline{document}, que señala dónde empieza y termina el documento.

\begin{latexcode}
\documentclass[article]
\begin{document}
¡Hola, \LaTeX!
\end{document}
\end{latexcode}

% \latexfile{mwe.tex}





\end{document}
