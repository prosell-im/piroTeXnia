\chapter{Un poquito de matemáticas}

\index{mate|(}
En la sección~\ref{sec:sublipsum} (pág.~\pageref{sec:sublipsum}), del capítulo~\ref{cap:lipsum}

\begin{equation*}
  e^{e^{e^e}}
\end{equation*}

\section{Enunciados distinguidos}

Por un \emph{enunciado distinguido}\index{enunciado distinguido}\index{enunciado!distinguido} entenderemos lo siguiente: teorema, lema,\index{ultima@última} proposición, corolario.

Uno define sus propios enunciados distinguidos, junto con la jerarquía y profundidad de su enumeración.

Los reales, $\R$, son bla, bla.

Una $\sigma$-álgebra\index{sigmaalgebra@$\sigma$-álgebra} es no se qué.

\subsection{Ejemplos}

\begin{teo}
  \leavevmode
  \begin{enumerate}
    \item Primer resultado.
    \item Segundo resultado.
  \end{enumerate}
Mi primer teorema.
\end{teo}

\begin{lema}\label{lema:uno}
Un primer lema.
\end{lema}

\begin{teo}[Cauchy]
  Mi segundo teorema.
  \end{teo}
\begin{proof}
  Mi demostración.
  \begin{equation*}
    x=\omega.\qedhere
  \end{equation*}
\end{proof}

\begin{defi}
  Una primera definición.
\end{defi}

\begin{obs}
Por el lema~\ref{lema:uno} 
\end{obs}

\section{Expresiones desplegadas}
Todos los desplegados serán del paquete \texttt{amsmath}.

Para expresiones \enquote{sencillas} se usará \texttt{equation}:

\begin{equation}
  e^{i\pi} + 1 = 0.
\end{equation}

\begin{equation*}
  e^{i\pi} + 1 = 0.
\end{equation*}

Para expresiones muuuy largas usamos \texttt{multline}

\begin{multline*}
  \sum_{i=0}^{\infty} \Delta_1(x_0)+\Delta_2(x_1) - \Gamma_1(x_0)-\Gamma_2(x_1) +
  \Delta_1(x_0)+\Delta_2(x_1)\\[-5pt]
  - \Gamma_1(x_0)-\Gamma_2(x_1) +
  \Delta_1(x_0)+\Delta_2(x_1) - \Gamma_1(x_0)-\Gamma_2(x_1)\\[3pt]
  - \Gamma_1(x_0)-\Gamma_2(x_1) +
  \Delta_1(x_0)+\Delta_2(x_1) - \Gamma_1(x_0)-\Gamma_2(x_1)\\[3pt]
  - \Gamma_1(x_0)-\Gamma_2(x_1) +
  \Delta_1(x_0)+\Delta_2(x_1) - \Gamma_1(x_0)-\Gamma_2(x_1)
\end{multline*}

\begin{multline*}
  \sum_{i=0}^{\infty} \Delta_1(x_0)+\Delta_2(x_1) - \Gamma_1(x_0)-\Gamma_2(x_1) +
  \Delta_1(x_0)+\Delta_2(x_1)\\
  - \Gamma_1(x_0)-\Gamma_2(x_1) +
  \Delta_1(x_0)+\Delta_2(x_1) - \Gamma_1(x_0)-\Gamma_2(x_1)\\
  - \Gamma_1(x_0)-\Gamma_2(x_1) +
  \Delta_1(x_0)+\Delta_2(x_1) - \Gamma_1(x_0)-\Gamma_2(x_1)\\
  - \Gamma_1(x_0)-\Gamma_2(x_1) +
  \Delta_1(x_0)+\Delta_2(x_1) - \Gamma_1(x_0)-\Gamma_2(x_1)
\end{multline*}

\begin{gather}
  x^2 = a^2 + b^2 \\
  \frac{a}{\sen A} =\frac{b}{\sen B} = \frac{c}{\sen C} \label{ec:leydesenos}
\end{gather}

La ecuación~\ref{ec:leydesenos}

\begin{equation}
\begin{gathered}
  x^2 = a^2 + b^2 \\
  \frac{a}{\sen A} =\frac{b}{\sen B} = \frac{c}{\sen C} 
\end{gathered}
\end{equation}

Para alinear ecuaciones usamos el entorno \texttt{align}

\begin{align}
  ax^2 + bx + c &= 0\\
  ax^2 &= -bx -c\\
  x -1 &< \frac{-b \pm\sqrt{b^2 - 4ac}}{2a}
\end{align}

\begin{align*}
  2a + 3b &= 10 & -3\alpha - 2\beta &= 9 & &\text{(primera ecuación)}\\
  -5a + 7b &= 7 & \alpha + \beta &= 8 & &\text{(segunda)}
\end{align*}

Para definir ecuaciones por casos usamos:

\begin{equation*}
  |f(x)| = \begin{cases}
    f(x) & \text{si $f(x) \geq 0$}\\
    -f(x) & \text{si $f(x) < 0$}
  \end{cases}
\end{equation*}

Para poner delimitadores a mi antojo:

\begin{equation*}
  \left|\begin{aligned}
    2a+3b &= 10\\
    -5a+7b &= 7
  \end{aligned}\right\} \qquad
  \left.\begin{aligned}
    2a+3b &= 10\\
    -5a+7b &= 7
  \end{aligned}\right\}
\end{equation*}

Muchas delimitadores para matrices:\index{mate!matrices}

\begin{gather*}
  \begin{matrix}
    a & b & c \\
    b & c & d
  \end{matrix}\qquad
  \begin{pmatrix}
    a & b & c \\
    b & c & d
  \end{pmatrix}\qquad
  \begin{bmatrix}
    a & b & c \\
    b & c & d
  \end{bmatrix}\qquad 
  \begin{Bmatrix}
    a & b & c \\
    b & c & d
  \end{Bmatrix}\\[10pt]
  \begin{vmatrix}
    a & b & c \\
    b & c & d
  \end{vmatrix}\qquad  
  \begin{Vmatrix}
    a & b & c \\
    b & c & d
  \end{Vmatrix}\qquad
\end{gather*}


\begin{equation*}
\binom{n}{r} = \frac{n!}{r!(n-r)!}\qquad
\Bigl(\sum_{i=0}^{\infty} x \Bigr)
\end{equation*}

La ley de senos en modo párrafo se ve re-fea: $\displaystyle{\frac{a}{\sen A} =\frac{b}{\sen B} = \frac{c}{\sen C}}$ donde $\sen A$ es el ángulo formado por los lados, bla bla y más texto para que se note lo feo. La suma $\sum_{i=0}^{\infty} x_i$ converge si, bla
\index{mate|)}

Para definir un operador de forma adecuada usamos \verb+\DeclareMathOperator+ para por ejemplo:
\begin{equation*}
  \Proj(x,y) \coloneq x\cdot y^\perp
\end{equation*}

\endinput