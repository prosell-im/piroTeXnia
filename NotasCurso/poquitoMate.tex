\chapter{Un poquito de matemáticas}

En la sección~\ref{sec:sublipsum} (pág.~\pageref{sec:sublipsum}), del capítulo~\ref{cap:lipsum}

\begin{equation*}
  e^{e^{e^e}}
\end{equation*}

\section{Enunciados distinguidos}

Por un \emph{enunciado distinguido}\index{enunciado distinguido}\index{enunciado!distinguido} entenderemos lo siguiente: teorema, lema,\index{\'ultima} proposición, corolario.

Uno define sus propios enunciados distinguidos, junto con la jerarquía y profundidad de su enumeración.

\subsection{Ejemplos}

\begin{teo}
  \leavevmode
  \begin{enumerate}
    \item Primer resultado.
    \item Segundo resultado.
  \end{enumerate}
Mi primer teorema.
\end{teo}

\begin{lema}\label{lema:uno}
Un primer lema.
\end{lema}

\begin{teo}[ Cauchy]
  Mi segundo teorema.
  \end{teo}
\begin{proof}
  Mi demostración.
  \begin{equation*}
    x=\omega.\qedhere
  \end{equation*}
\end{proof}

\begin{defi}
  Una primera definición.
\end{defi}

\begin{obs}
Por el lema~\ref{lema:uno} 
\end{obs}

\section{Expresiones desplegadas}

\endinput