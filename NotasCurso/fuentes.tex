\chapter{Elementos de texto}
\section{Fuentes y tipos}

\subsection*{Estilos}
\addcontentsline{toc}{section}{Estilos}

 El texto que dice \enquote{negritas} está en \textbf{negritas}, y también en {\bfseries negritas}.
 \index{enunciado distinguido}
 
 El texto que dice \enquote{cursivas} está en \textit{cursivas}, y también en {\itshape cursivas}
 
 El texto que dice \enquote{inclinadas} está en \textsl{inclinadas}, y también en {\slshape inclinadas}

 El texto que dice \enquote{sans serif} está en \textsf{sans serif}, y también en {\sffamily \textbf{sans serif}}

 El texto que dice \enquote{manoespaciada} está en \texttt{monoespaciada}, y también en {\ttfamily monoespaciada}

 {\ttfamily \textit{\textbf{monoespaciada}}}

Un párrafo que tenga el siguiente texto \emph{enfatizado}, ya no. Como se puede ver en la tabla~\ref{T:unatabla}, los, bla, bla

\textit{Un párrafo que tenga el siguiente texto \emph{enfatizado}, ya no.
}

\subsection{Tipos (tamaños)}

{\tiny Texto}
{\footnotesize Texto}{\small Texto}{\normalsize Texto}
{\large Texto}{\Large Texto}{\LARGE\bfseries\sffamily Texto}
{\huge Texto}{\Huge Texto}

% \begin{table}
%   Aquí hay una tabla
%   \caption{Una tabla}\label{T:unatabla}
% \end{table}

\section{Listas}

Listas enumeradas:
\begin{enumerate}
  \item Primer elemento.
  \item Segundo.
  \begin{itemize}
    \item Primer elemento.
    \item Segundo.
    \item etc. etc.
  \end{itemize}  
  \item etc. etc.
\end{enumerate}


\endinput
