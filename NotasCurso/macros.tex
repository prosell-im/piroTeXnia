% Mis enunciados distinguidos
\theoremstyle{plain}
\newtheorem{teo}{Teorema}[chapter]
\newtheorem{lema}[teo]{Lema}

\theoremstyle{definition}
\newtheorem{defi}{Definición}

\theoremstyle{remark}
\newtheorem{obs}{Observación}

% Mis comandos para este documento
\newcommand{\indice}[2][textbf]{#2\index{#2|#1}}
\newcommand{\R}{\ensuremath{\mathbb{R}}}
\DeclareMathOperator{\Proj}{Proj}

\newcommand\EatDot[1]{}

% Para poner las urls en una línea nueva.
\newbibmacro*{bbx:parunit}{%
  \ifbibliography
    {\setunit{\bibpagerefpunct}\newblock
     \usebibmacro{pageref}%
     \clearlist{pageref}%
     \setunit{\adddot\par\nobreak}
    }
  {}
}

\renewbibmacro*{doi+eprint+url}{%
  \usebibmacro{bbx:parunit}% Added
  \iftoggle{bbx:doi}
    {\printfield{doi}}{}%
  \iftoggle{bbx:eprint}
    {\usebibmacro{eprint}}{}%
  \iftoggle{bbx:url}
    {\usebibmacro{url+urldate}}{}%    
}

\renewbibmacro*{addendum+pubstate}{%
  \setunit{}
  \printfield{addendum}%
  \newunit\newblock
  \printfield{pubstate}}

% Formato para el addendum de biblatex 
\usepackage{mdframed}
\newcommand{\miaddendum}[1]{\begin{mdframed}[topline=false,bottomline=false,rightline=false,innertopmargin=0pt,innerbottommargin=-1.5pt,linecolor=gray,linewidth=2pt]\footnotesize #1\end{mdframed}}

\endinput