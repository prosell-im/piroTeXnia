\chapter{Entornos flotantes}
Por omisión \LaTeX\ ofrece dos tipos de \indice{entornos flotantes}. Uno para figuras y otro para tablas.

 La idea es que los acomoda donde mejor puede. Busca acomodarlos lo más cercano a donde se invocan ya sea hasta arriba o hasta abajo de la página. Este acomodo se puede modificar vía parámetros opcionales. 

 El paquete \texttt{float}\index{paquete!float@\texttt{float}} permite obligar a la mala a incluirlos exactamente donde se invocan. Además permite definir nuevos \emph{floats}.

 Ejemplos:

 \begin{figure}[htb]
  \centering
  \begin{overpic}[scale=1.3]{FiguraLucero.pdf}
    \small
    \put(20,65){$\Omega$}
    \put(30,0){$x_0$}
    \put(53,30){Lo que sea}
  \end{overpic}
  \caption{De lo que va la figura}\label{fig:primerafigura}
 \end{figure}

La figura~\ref*{fig:primerafigura} muestra el cono visible desde el punto $x_0$.

Para una tabla (cuadro en España) se usa así:

\begin{table}
%  \renewcommand{\arraystretch}{1.4}
  \centering
  \begin{tabular}{lcrp{3cm}}
    \toprule
    Izquierda & Centrado & Derecha & Párrafo \\ \midrule
    10 & 5 & 200 & Algo de texto para que se note.\\ 
    10 & 5 & 200 & Algo de texto para que se note.\\ \bottomrule
  \end{tabular} 
  \caption{}\label{tabla:unatabla}
\end{table}
