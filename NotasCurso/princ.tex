%\documentclass[openany]{book} PAra que los capítulos abran en página par o impar
\documentclass{book}

\usepackage{notas-temp}


% Mis enunciados distinguidos
\theoremstyle{plain}
\newtheorem{teo}{Teorema}[chapter]
\newtheorem{lema}[teo]{Lema}

\theoremstyle{definition}
\newtheorem{defi}{Definición}

\theoremstyle{remark}
\newtheorem{obs}{Observación}

% Mis comandos para este documento
\newcommand{\indice}[2][textbf]{#2\index{#2|#1}}
\newcommand{\R}{\ensuremath{\mathbb{R}}}
\DeclareMathOperator{\Proj}{Proj}

\newcommand\EatDot[1]{}

% Para poner las urls en una línea nueva.
\newbibmacro*{bbx:parunit}{%
  \ifbibliography
    {\setunit{\bibpagerefpunct}\newblock
     \usebibmacro{pageref}%
     \clearlist{pageref}%
     \setunit{\adddot\par\nobreak}
    }
  {}
}

\renewbibmacro*{doi+eprint+url}{%
  \usebibmacro{bbx:parunit}% Added
  \iftoggle{bbx:doi}
    {\printfield{doi}}{}%
  \iftoggle{bbx:eprint}
    {\usebibmacro{eprint}}{}%
  \iftoggle{bbx:url}
    {\usebibmacro{url+urldate}}{}%    
}

\renewbibmacro*{addendum+pubstate}{%
  \setunit{}
  \printfield{addendum}%
  \newunit\newblock
  \printfield{pubstate}}

% Formato para el addendum de biblatex 
\usepackage{mdframed}
\newcommand{\miaddendum}[1]{\begin{mdframed}[topline=false,bottomline=false,rightline=false,innertopmargin=0pt,innerbottommargin=-1.5pt,linecolor=gray,linewidth=2pt]\footnotesize #1\end{mdframed}}

\endinput

%%%% Para la bibliografía
\addbibresource{notas.bib}
\addbibresource{mate.bib}

\title{Notas de \LaTeX}
\author{Joseph\thanks{IM-UNAM}\and  Ernesto\thanks{Cimat} \and Jimena\thanks{FFyL-UNAM} \and Erick\thanks{FC-UNAM}}

\begin{document}
\frontmatter
%\maketitle
\addtocounter{page}{8}
\chapter{Prefacio}

Habla un poco de la estructura y agradecimientos.

\endinput
\tableofcontents
\listoffigures
\listoftables


\mainmatter
\chapter{Introducción}

Generalidades y prerequisitos

\endinput

No importa nada, bla, bla
\chapter{Elementos de texto}
\section{Fuentes y tipos}

\subsection*{Estilos}
\addcontentsline{toc}{section}{Estilos}

 El texto que dice \enquote{negritas} está en \textbf{negritas}, y también en {\bfseries negritas}.
 \index{enunciado distinguido}
 
 El texto que dice \enquote{cursivas} está en \textit{cursivas}, y también en {\itshape cursivas}
 
 El texto que dice \enquote{inclinadas} está en \textsl{inclinadas}, y también en {\slshape inclinadas}

 El texto que dice \enquote{sans serif} está en \textsf{sans serif}, y también en {\sffamily \textbf{sans serif}}

 El texto que dice \enquote{monoespaciada} está en \texttt{monoespaciada}, y también en {\ttfamily monoespaciada}

 {\ttfamily \textit{\textbf{monoespaciada}}}

Un párrafo que tenga el siguiente texto \emph{enfatizado}, ya no. Como se puede ver en la tabla~\ref{tabla:unatabla}, los, bla, bla
Un párrafo que tenga el siguiente texto \emph{enfatizado}, ya no. Como se puede ver en la tabla, los, bla, bla

\textit{Un párrafo que tenga el siguiente texto \emph{enfatizado}, ya no.
}

\subsection{Tipos (tamaños)}

{\tiny Texto}
{\footnotesize Texto}{\small Texto}{\normalsize Texto}
{\large Texto}{\Large Texto}{\LARGE\bfseries\sffamily Texto}
{\huge Texto}{\Huge Texto}

% \begin{table}
%   Aquí hay una tabla
%   \caption{Una tabla}\label{T:unatabla}
% \end{table}

\section{Listas}

Listas enumeradas:
\begin{enumerate}
  \item Primer elemento.
  \item Segundo.
  \begin{itemize}
    \item Primer elemento.
    \item Segundo.
    \item etc. etc.
  \end{itemize}  
  \item etc. etc.
\end{enumerate}


\endinput

\printbibliography[title=Referencias, heading=subbibintoc, notkeyword=mate]
\chapter{Un poquito de matemáticas}

\index{mate|(}
En la sección~\ref{sec:sublipsum} (pág.~\pageref{sec:sublipsum}), del capítulo~\ref{cap:lipsum}

\begin{equation*}
  e^{e^{e^e}}
\end{equation*}

\section{Enunciados distinguidos}

Por un \emph{enunciado distinguido}\index{enunciado distinguido}\index{enunciado!distinguido} entenderemos lo siguiente: teorema, lema,\index{ultima@última} proposición, corolario.

Uno define sus propios enunciados distinguidos, junto con la jerarquía y profundidad de su enumeración.

Los reales, $\R$, son bla, bla.

Una $\sigma$-álgebra\index{sigmaalgebra@$\sigma$-álgebra} es no se qué.

\subsection{Ejemplos}

\begin{teo}
  \leavevmode
  \begin{enumerate}
    \item Primer resultado.
    \item Segundo resultado.
  \end{enumerate}
Mi primer teorema.
\end{teo}

\begin{lema}\label{lema:uno}
Un primer lema.
\end{lema}

\begin{teo}[Cauchy]
  Mi segundo teorema.
  \end{teo}
\begin{proof}
  Mi demostración.
  \begin{equation*}
    x=\omega.\qedhere
  \end{equation*}
\end{proof}

\begin{defi}
  Una primera definición.
\end{defi}

\begin{obs}
Por el lema~\ref{lema:uno} 
\end{obs}

\section{Expresiones desplegadas}
Todos los desplegados serán del paquete \texttt{amsmath}.

Para expresiones \enquote{sencillas} se usará \texttt{equation}:

\begin{equation}
  e^{i\pi} + 1 = 0.
\end{equation}

\begin{equation*}
  e^{i\pi} + 1 = 0.
\end{equation*}

Para expresiones muuuy largas usamos \texttt{multline}

\begin{multline*}
  \sum_{i=0}^{\infty} \Delta_1(x_0)+\Delta_2(x_1) - \Gamma_1(x_0)-\Gamma_2(x_1) +
  \Delta_1(x_0)+\Delta_2(x_1)\\[-5pt]
  - \Gamma_1(x_0)-\Gamma_2(x_1) +
  \Delta_1(x_0)+\Delta_2(x_1) - \Gamma_1(x_0)-\Gamma_2(x_1)\\[3pt]
  - \Gamma_1(x_0)-\Gamma_2(x_1) +
  \Delta_1(x_0)+\Delta_2(x_1) - \Gamma_1(x_0)-\Gamma_2(x_1)\\[3pt]
  - \Gamma_1(x_0)-\Gamma_2(x_1) +
  \Delta_1(x_0)+\Delta_2(x_1) - \Gamma_1(x_0)-\Gamma_2(x_1)
\end{multline*}

\begin{multline*}
  \sum_{i=0}^{\infty} \Delta_1(x_0)+\Delta_2(x_1) - \Gamma_1(x_0)-\Gamma_2(x_1) +
  \Delta_1(x_0)+\Delta_2(x_1)\\
  - \Gamma_1(x_0)-\Gamma_2(x_1) +
  \Delta_1(x_0)+\Delta_2(x_1) - \Gamma_1(x_0)-\Gamma_2(x_1)\\
  - \Gamma_1(x_0)-\Gamma_2(x_1) +
  \Delta_1(x_0)+\Delta_2(x_1) - \Gamma_1(x_0)-\Gamma_2(x_1)\\
  - \Gamma_1(x_0)-\Gamma_2(x_1) +
  \Delta_1(x_0)+\Delta_2(x_1) - \Gamma_1(x_0)-\Gamma_2(x_1)
\end{multline*}

\begin{gather}
  x^2 = a^2 + b^2 \\
  \frac{a}{\sen A} =\frac{b}{\sen B} = \frac{c}{\sen C} \label{ec:leydesenos}
\end{gather}

La ecuación~\ref{ec:leydesenos}

\begin{equation}
\begin{gathered}
  x^2 = a^2 + b^2 \\
  \frac{a}{\sen A} =\frac{b}{\sen B} = \frac{c}{\sen C} 
\end{gathered}
\end{equation}

Para alinear ecuaciones usamos el entorno \texttt{align}

\begin{align}
  ax^2 + bx + c &= 0\\
  ax^2 &= -bx -c\\
  x -1 &< \frac{-b \pm\sqrt{b^2 - 4ac}}{2a}
\end{align}

\begin{align*}
  2a + 3b &= 10 & -3\alpha - 2\beta &= 9 & &\text{(primera ecuación)}\\
  -5a + 7b &= 7 & \alpha + \beta &= 8 & &\text{(segunda)}
\end{align*}

Para definir ecuaciones por casos usamos:

\begin{equation*}
  |f(x)| = \begin{cases}
    f(x) & \text{si $f(x) \geq 0$}\\
    -f(x) & \text{si $f(x) < 0$}
  \end{cases}
\end{equation*}

Para poner delimitadores a mi antojo:

\begin{equation*}
  \left|\begin{aligned}
    2a+3b &= 10\\
    -5a+7b &= 7
  \end{aligned}\right\} \qquad
  \left.\begin{aligned}
    2a+3b &= 10\\
    -5a+7b &= 7
  \end{aligned}\right\}
\end{equation*}

Muchas delimitadores para matrices:\index{mate!matrices}

\begin{gather*}
  \begin{matrix}
    a & b & c \\
    b & c & d
  \end{matrix}\qquad
  \begin{pmatrix}
    a & b & c \\
    b & c & d
  \end{pmatrix}\qquad
  \begin{bmatrix}
    a & b & c \\
    b & c & d
  \end{bmatrix}\qquad 
  \begin{Bmatrix}
    a & b & c \\
    b & c & d
  \end{Bmatrix}\\[10pt]
  \begin{vmatrix}
    a & b & c \\
    b & c & d
  \end{vmatrix}\qquad  
  \begin{Vmatrix}
    a & b & c \\
    b & c & d
  \end{Vmatrix}\qquad
\end{gather*}


\begin{equation*}
\binom{n}{r} = \frac{n!}{r!(n-r)!}\qquad
\Bigl(\sum_{i=0}^{\infty} x \Bigr)
\end{equation*}

La ley de senos en modo párrafo se ve re-fea: $\displaystyle{\frac{a}{\sen A} =\frac{b}{\sen B} = \frac{c}{\sen C}}$ donde $\sen A$ es el ángulo formado por los lados, bla bla y más texto para que se note lo feo. La suma $\sum_{i=0}^{\infty} x_i$ converge si, bla
\index{mate|)}

Para definir un operador de forma adecuada usamos \verb+\DeclareMathOperator+ para por ejemplo:
\begin{equation*}
  \Proj(x,y) \coloneq x\cdot y^\perp
\end{equation*}

\LaTeX{} can now compute: $ \frac{\sin (3.5)}{2} + 2\cdot 10^{-3}
  = \fpeval{sin(3.5)/2 + 2e-3} $, and $\sqrt{2} \approx \fpeval{trunc(sqrt(2),6)}$.

\endinput
\printbibliography[title=Referencias, heading=subbibintoc, keyword={mate}]
\chapter{Entornos flotantes}
Por omisión \LaTeX\ ofrece dos tipos de \indice{entornos flotantes}. Uno para figuras y otro para tablas.

 La idea es que los acomoda donde mejor puede. Busca acomodarlos lo más cercano a donde se invocan ya sea hasta arriba o hasta abajo de la página. Este acomodo se puede modificar vía parámetros opcionales. 

 El paquete \texttt{float}\index{paquete!float@\texttt{float}} permite obligar a la mala a incluirlos exactamente donde se invocan. Además permite definir nuevos \emph{floats}.

 Ejemplos:

 \begin{figure}[htb]
  \centering
  \begin{overpic}[scale=1.3]{FiguraLucero.pdf}
    \small
    \put(20,65){$\Omega$}
    \put(30,0){$x_0$}
    \put(53,30){Lo que sea}
  \end{overpic}
  \caption{De lo que va la figura}\label{fig:primerafigura}
 \end{figure}

La figura~\ref*{fig:primerafigura} muestra el cono visible desde el punto $x_0$.

Para una tabla (cuadro en España) se usa así:

\begin{table}
%  \renewcommand{\arraystretch}{1.4}
  \centering
  \begin{tabular}{lcrp{3cm}}
    \toprule
    Izquierda & Centrado & Derecha & Párrafo \\ \midrule
    10 & 5 & 200 & Algo de texto para que se note.\\ 
    10 & 5 & 200 & Algo de texto para que se note.\\ \bottomrule
  \end{tabular} 
  \caption{}\label{tabla:unatabla}
\end{table}

\chapter{Bibliografía y referencias}
Para generar referencias de formato flexible se recomienda de acuerdo con Grätzer~\cite{gratzer} usar \texttt{biblatex} que es una versión mucho más poderosa del original \texttt{bibtex} (ver~\cite{lamport}).

\chapter{Lipsum}\label{cap:lipsum}
\enquote{Una frase citada}
\lipsum
\section{Sublipsum}\label{sec:sublipsum}
\lipsum

\endinput

\appendix
\chapter{Lógica y conjuntos}
\chapter{Números complejos}
\nocite{*}
\backmatter
%\chapter{Bibliografía}
%\cleardoublepage

\defbibnote{texref}{En este apartado sólo se muestran referencias relacionadas a \TeX\ estricto.}
\printbibliography[heading=bibintoc, notkeyword=doc]
\printbibliography[heading=bibintoc, keyword=tex, title={Bibliografía sólo de \TeX}, prenote=texref]
\printbibliography[heading=bibintoc, keyword=doc, title={Manuales de los paquetes}]
%\addcontentsline{toc}{chapter}{Índice alfabético}
% No hace falta usando intoc en el paquete imakeidx
\printindex

\end{document}